\section{Introduction}
\label{sec:intro}
In social networks, a natural question to consider is how ideas and information may spread through them. Motivated by applications to viral marketing, \citet{domingos2001mining} considered this problem 
from the perspective of finding an initial set of nodes that could be targeted (for example with free samples) such that these nodes would spread the influence of a product throughout the network. Influence would be spread through a \textquote{word of mouth} effect where the initial nodes would recommend product adoption to their connections, who recommend it to their connections, and so on. 
This would cause a cascade of adoption in the network, with the ultimate goal of achieving the largest expected number of influenced nodes at the end of the process. 
This type of problem formulation has applications beyond marketing, such as in the study of how infectious diseases spread.  

\subsection{Influence Maximization}
\citet{kempe2003maximizing, kempe2005influential} later formalized this model as a discrete optimization problem concerned with the spread of influence in a network.
Given some stochastic model of diffusion through a network and a node set $A$, the number of active (i.e.\! influenced) nodes at the end of a diffusion cascade started by $A$ is denoted by $\func{\varphi}{A}$. 
Note that this is a random variable as the diffusion model is probabilistic. The \emph{influence} of $A$, denoted by $\infl{A}$, is then the expected value of the number of active nodes at the end of the diffusion process started by $A$ (i.e.\! $\infl{A} = \EX{\func{\varphi}{A}}$).
The \textsf{Influence Maximization (InfMax)} problem looks to find, for a parameter $k$, a \textquote{seed} set of nodes of size $k$ that maximizes influence. 
Under most diffusion models, this problem is $\NPH$ \cite{kempe2003maximizing, kempe2005influential}. 

\begin{problem}[\textsf{InfMax}]
    Given a digraph $G = \lrp{V, E}$, a model for diffusion, and an integer $k$, find a seed set $S \subseteq V$ of size $k$
    such that $\infl{S}$ is maximized. 
    \label{prob:infmax}
\end{problem}

\subsection{Diffusion Models}
For the purposes of a diffusion model, nodes in the network can either be active or inactive. These models are typically progressive, meaning that once a node becomes active, it cannot turn inactive. Denote the set of in-neighbors of a node $v$ by $\inneigh{v}$. Two common types of diffusion models are described below. 

\begin{definition}[LT Model]
    \label{def:linear_threshold}
    The \emph{Linear Threshold} model is defined by a digraph $G = (V, E)$ and an edge weight function $w: E \rightarrow (0, 1]$, where each node $v \in V$ is associated with some threshold $\theta_v$ and the weight of an edge $\lrp{u, v} \in E$ denotes the degree of influence $u$ has on $v$.
    Notably, $\theta_v$ is sampled uniformly at random from $\lrb{0, 1}$ and $\sum_{\lrc{u \;\mid\; u \in \Gamma \lrp{v}}} \func{w}{u,v} \leq 1$ for each $v \in V$. On an input set $S$, the model works as follows:
    \begin{enumerate}
        \item In time step $0$, the nodes in $S$ become activated
        \item In time step $t$, a node $v$ becomes activated if 
            \begin{equation*}
                \sum_{\lrc{u \;\mid\; u \in \Gamma(v) \text{ and } u \text{ is active}}} \func{w}{u,v} \geq \theta_v
            \end{equation*}
        \item The process ends after there is a time step with no additional activated nodes
    \end{enumerate}
    This process takes at most $\abs{V}$ time steps. 
\end{definition}

\begin{definition}[IC Model]
    \label{def:independent_cascade}
    The \emph{Independent Cascade} model is defined by a digraph $G = (V, E)$ and an edge weight function $p: E \rightarrow (0, 1]$, where the weight of each edge $\lrp{u,v} \in E$ represents the probability that $u$ activates $v$. 
    On an input set $S$, the model works as follows:
    \begin{enumerate}
        \item In time step $0$, the nodes in $S$ become activated
        \item In time step $t$, each newly activated node $u$ in time step $t-1$ gets one chance to activate each inactivated out-neighbor $v$ with probability $\func{p}{u,v}$
        \item The process ends after there is a time step with no additional activated nodes
    \end{enumerate}
    This process takes at most $D$ time steps, where $D$ is the diameter of $G$.
\end{definition}