\documentclass[11pt]{article}
\usepackage[biblatex]{preamble}
\addbibresource{refs.bib}

\newcommand{\infmax}{\hyperref[prob:infmax]{\textsf{InfMax}}}
\newcommand{\infl}[1]{\ensuremath{\func{\sigma}{#1}}}
\newcommand{\inneigh}[1]{\ensuremath{\func{\Gamma}{#1}}}
\newcommand{\ltmodel}[1][LT]{\hyperref[def:linear_threshold]{\textsf{#1}}}
\newcommand{\icmodel}[1][IC]{\hyperref[def:independent_cascade]{\textsf{#1}}}

\title{Notes on Influence Maximization}
%\title{Influence Maximization while Exploring a Network}
%\author{Bhargav Samineni}
\date{}

\begin{document}

\maketitle

\section{Prior Work}
\customcite{eshghi2019efficient}
\begin{problem}[PV-IM]
    Suppose there exists some underlying directed graph $G = \lrp{V, E}$ where $\abs{V} = n, \abs{E} = m$. If we are given
    \begin{itemize}[label=--]
        \itemsep0em 
        \item a visible induced subgraph of this network $G_v = \lrp{V_v, E_v}$ where $V_v \sse V$
        %and $E_v = \lrc{e = \lrp{a, b} \in E \colon a, b \in V_v}$
        \item the structure of the unobserved part of the network is one of $M$ different realizations $G_1, \ldots, G_M$
        each with respective probability $q_i$ such that $\sum_{i=1}^{M} q_i = 1$ 
        \item a diffusion model (either Independent Cascade or Weighted Cascade) with edge weight function $p \colon E \rightarrow \lrb{0, 1}$ and positive integer $k$
    \end{itemize}  
    find a seed set $S \sse V_v$ of size $k$ that maximizes $\infl{S}$. This is defined to be the a priori expected spread from $S$ among all 
    realizations (i.e.\! $\infl{S} = \sum_{i=1}^{M} q_i \f[\sigma_i]{S}$ where $\f[\sigma_i]{S}$ is the expected spread of $S$ in the graph $\calG_i = G_v \cup G_i$). 
\end{problem}

The main contribution of this paper is to extend the idea of Reverse Reachable (RR) sets to this new setting by specifying 
how many RR sets need to be generated for each network realization $\calG_i$.
Let $\f[x_i^j]{S}$ be a binary variable 
denoting whether the $j$th RR set for $\calG_i$ overlaps with any of the nodes in set $S$. Based on this we can define 
\begin{equation*}
    \f[F_i]{S} = \frac{1}{\theta_i} \sum_{j=1}^{\theta_i} \f[x_i^j]{S}
\end{equation*}
to be the fraction of RR sets generated on $\calG_i$ that $S$ covers. This value is an unbiased estimator of $\f[\sigma_i]{S} / n$, which gives us that 
$\f[F]{S} = \sum_{i=1}^{M} q_i \f[F_i]{S}$ (i.e.\! the weighted average of the fraction of total RR sets that overlap with $S$) is an unbiased estimator of 
$\f[\sigma]{S} / n$. 
 

Let $S^{\ast}$ be an optimal seed set. Define $p^{\ast}$ to be the expected influenced fraction of nodes in the network with $S^{\ast}$ as a seed set 
(i.e.\! $p^{\ast} = \f[\sigma]{S^{\ast}} / n$). For each graph realization $\calG_i$, define $p_i^{\ast}$ similarly (i.e.\! $p_i^{\ast} = \f[\sigma_i]{S} / n$). 
Then $p^{\ast} = \sum_{i=1}^{M} q_i p_i^{\ast}$. 

\begin{lemma}[\cite{eshghi2019efficient}, Lemma 1]
    For $\delta_1 \in \lrp{0, 1}$ and $\varepsilon_1 > 0$, if 
    \begin{equation}
        \theta^{\ast} = \frac{\sum_{i=1}^{M} q_i^2 p_i^{\ast} \lrp{1 - p_i^{\ast}}}{\lrp{\sum_{i=1}^{M} q_i p_i^{\ast}}^2} \frac{\f[\log]{1 / \delta_1}}{\varepsilon_1^2}
        = \frac{\sum_{i=1}^{M} q_i^2 p_i^{\ast} \lrp{1 - p_i^{\ast}}}{\lrp{p^{\ast}}^2} \frac{\f[\log]{1 / \delta_1}}{\varepsilon_1^2},
        \label{eq:theta_star}
    \end{equation}
    then for $\theta \geq \theta^{\ast}$, $n\f[F]{S^{\ast}} \geq \lrp{1 - \varepsilon_1}\f[\sigma]{S^{\ast}}$ with probability at least $1 - \delta_1$, where each realization $\calG_i$ is sampled 
    \begin{equation}
        \theta_i^{\ast} = \theta^{\ast} \frac{q_i \sqrt{p_i^{\ast} \lrp{1 - p_i^{\ast}} }}{\sum_{j=1}^{M} q_j \sqrt{p_j^{\ast} \lrp{1 - p_j^{\ast}} }}
        \label{eq:theta_star_i}
    \end{equation}
    times for RR sets. 
    \label{lem:theta_star}
\end{lemma}

\cref{lem:theta_star} implies that if a set $S$ is constructed greedily from nodes covering these $\theta^{\ast}$ RR sets, then 
\begin{equation*}
    n\f[F]{S} \geq \lrp{1 - \frac{1}{e}} n \f[F]{S^{\ast}} \geq \lrp{1 - \frac{1}{e}} \lrp{1 - \varepsilon_1} \f[\sigma]{S^*} = \lrp{1 - \frac{1}{e} - \varepsilon_1 \lrp{1 - \frac{1}{e}}} \f[\sigma]{S^{\ast}}
\end{equation*}
with probability $1 - \delta_1$. Intuitively, since $n \f[F]{S}$ is an indicator of $\f[\sigma]{S}$, this implies that $S$ is likely to be large. 
However, the greedy algorithm can still construct a suboptimal seed set with some probability. 
To bound this possibility, we may need to generate more RR sets to ensure that the estimators of these suboptimal seed sets are 
close to their expected values. 

\begin{lemma}
    For $\delta_2 \in \lrp{0, 1}$, $\varepsilon > \varepsilon_1 \lrp{1 - \frac{1}{e}} > 0$, if \cref{lem:theta_star}
    holds and 
    \begin{equation}
        \theta^\prime = \frac{2 \f[\log]{\frac{\binom{n}{k}}{\delta_2}} \lrb{\lrp{1 - \varepsilon_1}\lrp{1 - \frac{1}{e}} - \frac{2}{3}\varepsilon}}{\lrp{\varepsilon - \varepsilon_1 \lrp{1 - \frac{1}{e}}}^2 p^*},
        \label{eq:theta_prime_i}
    \end{equation}
    then for $\theta \geq \theta^\prime$, $\f[\sigma]{S} \geq \lrp{1 - \frac{1}{e} - \varepsilon} \f[\sigma]{S^*}$
    with probability at least $1 - \delta_2$, where each realization $\calG_i$ is sampled $\theta_i^\prime = q_i \theta^\prime$ for RR sets. 
    \label{lem:theta_prime}
\end{lemma}

\cref{lem:theta_star,lem:theta_prime} then give the following main result. 

\begin{theorem}
    For $\varepsilon > \varepsilon_1 \lrp{1 - \frac{1}{e}} > 0$, $\delta_1 , \delta_2 \in \lrp{0, 1}$ and $\delta = \delta_1 + \delta_2$, 
    if $\calG_i$ is sampled $\theta_i = \max \lrc{\theta_i^{\ast}, \theta_i^\prime}$ times for RR sets for all $i$, 
    then the greedy algorithm returns a $\lrp{1 - \frac{1}{e} - \varepsilon}$ approximation with probability at least $1 - \delta$.     
    \label{thm:}
\end{theorem}

\section{Extensions}

The actual construction of the RR sets do not necessarily depend on the diffusion model used. As long as there is a way to 
create the RR sets, then the same bounds on the number of RR sets needed for each $\calG_i$ and the general algorithm of \cite{eshghi2019efficient} can still be applied. 
\cite{tang2015influence} show that the triggering model still allows for the generation of RR sets. In general, if the diffusion model used 
does not admit a submodular influence spread function, then it is impossible to generate RR sets under that model. 

In the original IMM paper \cite{tang2015influence}, they provide a lower bound on the size of $\f[\sigma]{S^*}$ as that value is used 
in constructing the bounds on the number of RR sets needed. This paper does not provide that analysis. 

\begin{comment}
The main contribution of this paper is to extend the idea of Reverse Reachable (RR) sets to this new setting. If we denote 
$\calG_i = G_v \cup G_i$ for an unobserved realization $i$ and $S$ a seed set, then we can define some unbiased estimator of $\infl{S}$ based on the weighted average of the proportion 
of random RR sets generated for each $\calG_i$ that overlap with $S$. We want this estimator to be close enough to $\infl{S}$ so that a simple greedy algorithm that iteratively adds nodes 
chooses a seed set within an acceptable range with high probability. This can be accomplished in two steps 
\begin{enumerate}
    \item Choosing enough RR sets such that for an optimal seed set $S^{\ast}$, the estimator of $\infl{S^*}$ is close with high probability
    \item Choosing more RR sets (if needed) to ensure that the influence estimators of suboptimal seed sets will also be close to their expected vale. 
\end{enumerate}

The paper gives lower bounds on the number of RR sets needed to be chosen for each graph realization $\calG_i$ such that the greedy 
algorithm gives a $\lrp{1 - \frac{1}{e} - \varepsilon}$ approximation with high probability.  
\end{comment}

\newpage
\printbibliography

\end{document}